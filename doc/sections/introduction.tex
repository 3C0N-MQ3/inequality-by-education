\begin{frame}

    \begin{block}{Research Question}
        What is the impact of a substantial change in the share of the population with no more than 12 years of schooling on between-education inequality?
    \end{block}
    
    \begin{block}{Methodological Approach}
        The exercise employs a difference-in-differences (diff-in-diff) framework, using the change in the share of the population with less than 12 years of education as the endogenous variable, instrumented using the Mexican population growth between 1990 and 2000 in different regions. Key controls are derived from 1990 data.
    \end{block}
    
    \begin{block}{Key Findings}
        \begin{itemize}
            \item No significant effects were observed on differences in wage growth rates between highly educated and less-educated workers.
            \item Unemployment rates for highly educated workers increased significantly more than for less-educated workers.
            \item Rates of labor force inactivity (NILF) grew significantly more for highly educated workers compared to less-educated workers.
        \end{itemize}
    \end{block}

\end{frame}
