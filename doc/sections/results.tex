\begin{frame}
    \frametitle{Main Results}
    \framesubtitle{Instrumental Relevance}

    \begin{table}[ht]
        \centering
        \caption{First-Stage 2SLS Results}
        \label{tab:fs}
        \resizebox{0.65\textwidth}{!}{
            \begin{tabular}{rcccccc}
                \toprule
                \multicolumn{3}{c}{Endogenous Variable: Change in the share of the population with less than 12 years of education} \\
                \cmidrule(lr){2-3}
                                                                              & (1)             & (2)             \\
                \cmidrule(lr){1-1} \cmidrule(lr){2-3}
                \bf{Partial F Statistic}                                      & 22.376          & 44.856          \\
                \bf{P-value}                                                  & 0.0000          & 0.0000          \\
                \cmidrule(lr){1-1} \cmidrule(lr){2-3} 
                Constant                                                      & -0.0362***      & -0.2737***      \\
                                                                              & \small(0.0027)  & \small(0.0311)  \\
                Share of the college-educated population in 1990              &                 & 0.0130          \\
                                                                              &                 & \small(0.0229)  \\
                Share of employment among the foreign-born population 1990    &                 & 0.0607***       \\
                                                                              &                 & \small(0.0119)  \\
                Share of employment among women in 1990                       &                 & 0.0353          \\
                                                                              &                 & \small(0.0775)  \\
                Share of population with high school education in 1990        &                 & 0.2617***       \\
                                                                              &                 & \small(0.0170)  \\
                Logarithm of total population in 1990                         &                 & 0.0000          \\
                                                                              &                 & \small(0.0013)  \\
                Share of employment in manufacturing in 1990                  &                 & -0.0431***      \\
                                                                              &                 & \small(0.0139)  \\
                Instrument                                                    & 0.2015***       & 0.2276***       \\
                                                                              & \small(0.0426)  & \small(0.0340)  \\
                \cmidrule(lr){1-1} \cmidrule(lr){2-3}
                %\midrule
                $R^2$                                                         & 0.1834          & 0.7092          \\
                Partial $R^2$                                                 & 0.1834          & 0.2424          \\
                \bottomrule
            \end{tabular}
        }
        \vspace{0.2cm}
        
        \begin{minipage}{\linewidth}
            \tiny
            \textit{Notes:} $N = 741$. Robust standard errors (in parentheses) are clustered at the state level. A Partial F-statistic below 10 is typically considered weak evidence of instrument relevance. The p-value is calculated using a $\chi^2(1)$ distribution. Weighted regression based on the population size in each "CONSPUMA."\\
            *** Significant at the 1 percent level. \\
            ** Significant at the 5 percent level. \\
            * Significant at the 10 percent level.
        \end{minipage}
    \end{table}


\end{frame}
%%%%%%%%%%%%%%%%%%%%%%%%%%%%%%%%%%%%%%%%%%%%%%%%%%%%%%%%%%%%%%%%%%%%%%%%%%%
\begin{frame}
    \frametitle{Main Results}
    \framesubtitle{Instrumental Relevance}

    \begin{block}

        The results in Table \ref{tab:fs} present the partial $F$-statistic corresponding to the first stage of the 2SLS regression. According to the commonly accepted rule by Staiger and Stock (1997), and considering that the value of the partial $F$-statistic exceeds 10 in both exercises, it is concluded that the instrument is relevant for the measurement of the change in the proportion of individuals with less than 12 years of schooling.    \end{block}
\end{frame}
%%%%%%%%%%%%%%%%%%%%%%%%%%%%%%%%%%%%%%%%%%%%%%%%%%%%%%%%%%%%%%%%%%%%%%%%%%%
\begin{frame}
    \frametitle{Main Results}
    \framesubtitle{Causal Effect on Native Population Outcomes}

    \begin{table}[ht]
        \centering
        \caption{Results of the 2SLS on the Difference Between High and Low Education Levels for Various Outcomes of the Native Population}
        \label{tab:native_outcomes}
        \resizebox{1\textwidth}{!}{
            \begin{tabular}{rcccccccccccc}
                \toprule
    \multicolumn{1}{r}{\bf{Native Population Outcome}} & \multicolumn{2}{c}{Growth Rate of Wages} & \multicolumn{2}{c}{Growth Rate of Unemployment} & \multicolumn{2}{c}{Growth Rate of NILF} \\
                \cmidrule(lr){2-3} \cmidrule(lr){4-5} \cmidrule(lr){6-7}
                    & (1)         & (2)        & (3)         & (4)        & (5)        & (6)\\
                    \cmidrule(lr){1-3} \cmidrule(lr){4-5} \cmidrule(lr){6-7}
     
    
    Constant Term
    
                    & -0.010***     & 0.120*        & 0.163***      & -0.047        & 0.0122***       &  0.0749         \\
                    & \small(0.002) & \small(0.068) & \small(0.017) & \small(0.198) & \small(0.0020)  &  \small(0.0516) \\

    Share of the college-educated population in 1990             
    
                    &             & -0.012        &             & 0.254**       &                & -0.2459***     \\
                    &             & \small(0.043) &             & \small(0.129) &                & \small(0.0522) \\
    
    Share of employment among the foreign-born population in 1990   
    
                    &             & -0.076*       &             & 0.200**       &                & -0.0692**      \\
                    &             & \small(0.039) &             & \small(0.079) &                & \small(0.0275) \\
    
    Share of employment among women in 1990                      
    
                    &             & -0.353***     &             & -0.225        &                & 0.1914*        \\
                    &             & \small(0.080) &             & \small(0.354) &                & \small(0.1096) \\
    
    Share of population with high school education in 1990                      
    
                    &             & 0.011         &             & 0.230*        &                & -0.0434        \\
                    &             & \small(0.063) &             & \small(0.125) &                & \small(0.0511) \\
    
    Logarithm of total population in 1990                        
    
                    &             & 0.002        &             & -0.002         &                & -0.0033**      \\
                    &             & \small(0.001) &             & \small(0.005) &                & \small(0.0017) \\
    
    Share of employment in manufacturing in 1990                 
    
                    &             & 0.044*       &             & 0.328***       &                & 0.0225         \\
                    &             & \small(0.025) &             & \small(0.093) &                & \small(0.0273) \\
     
                \cmidrule(lr){2-3} \cmidrule(lr){4-5} \cmidrule(lr){6-7}
    
     \bf{Change in the share of the population with less than 12 years of education}
    
                    & -0.074       & 0.154       & 1.902***    & 1.221**            & 0.4093***      & 0.9830***      \\
                    & \small(0.083) & \small(0.207) & \small(0.503) & \small(0.518) & \small(0.1279) & \small(0.1379) \\
                \bottomrule
                \end{tabular}
            }
            \vspace{0.2cm}
            
            \begin{minipage}{\linewidth}
                \tiny
                \textit{Notes:} $N = 741$. Robust standard errors (in parentheses) are clustered at the state level. Weighted regression based on the population size in each CONSPUMA.\\
                *** Significant at the 1 percent level. \\
                ** Significant at the 5 percent level. \\
                * Significant at the 10 percent level.
            \end{minipage}
    \end{table}
    

\end{frame}

%%%%%%%%%%%%%%%%%%%%%%%%%%%%%%%%%%%%%%%%%%%%%%%%%%%%%%%%%%%%%%%%%%%%%%%%%%%
\begin{frame}
    \frametitle{Main Results}
    \framesubtitle{Causal Effect on Native Population Outcomes}

    The results in Table \ref{tab:native_outcomes} show that, given an increase in the change in the share of the population with less than 12 years of education ($x$):

    \begin{itemize}
    \item There is no significant effect on the relative difference between the growth rates of wages for highly-educated and low-educated native workers (Columns 1 and 2).
    \item The growth in the unemployment rate of highly-educated native workers is significantly greater than that of low-educated workers (Column 4).
    \item The growth in the labor inactivity rate of highly-educated native workers is significantly greater than that of low-educated native workers (Column 6).
    \end{itemize}
\end{frame}

%%%%%%%%%%%%%%%%%%%%%%%%%%%%%%%%%%%%%%%%%%%%%%%%%%%%%%%%%%%%%%%%%%%%%%%%%%%
\begin{frame}
    \frametitle{Main Results}
    \framesubtitle{Causal Effect on Native Population Outcomes}

    Defining a region as highly exposed if it ranks in the 75th percentile \footnote{Percentiles weighted by population size.} of the change in population share with less than 12 years of education, while a low-exposure region corresponds to the 25th percentile:

    \begin{itemize}
        \item The unemployment growth rate for highly educated native workers relative to less-educated native workers in highly exposed regions is projected to be 4.58 percentage points higher than in low-exposure regions.

        \item The NILF growth rate for highly educated native workers relative to less-educated native workers in highly exposed regions is projected to be 3.68 percentage points higher than in low-exposure regions.
    \end{itemize}
\end{frame}

