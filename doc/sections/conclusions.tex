\begin{frame}
    \frametitle{Conclusions}
    \begin{itemize}
        \item \textbf{Instrument Relevance:} The instrument for the change in the population with less than 12 years of education is relevant, as the Partial F-statistics far exceed the threshold of 10.

        \vspace{0.3cm}
    
        \item \textbf{Main Findings:}
        \begin{itemize}
            \item \textit{Wage Growth:} No significant effects on the differences in wage growth rates between highly educated and less-educated workers were observed given an increase in the low-education population.
            \item \textit{Unemployment Growth:} The unemployment rate for highly educated workers increased significantly more than for less-educated workers following an increase in the low-education population.
            \item \textit{Inactivity Growth:} Rates of labor force inactivity (NILF) grew significantly more for highly educated workers than for less-educated workers following an increase in the low-education population.
        \end{itemize}
        \vspace{0.3cm}
    
        \item \textbf{Regional Disparities:}
        \begin{itemize}
            \item Highly exposed regions (top 75th percentile of changes in low-education population) show unemployment growth for highly educated workers exceeding that of less-educated workers by 4.58 percentage points.
            \item Inactivity rates for highly educated workers in these regions exceeded those of less-educated workers by 3.68 percentage points.
        \end{itemize}
    \end{itemize}
    \end{frame}
